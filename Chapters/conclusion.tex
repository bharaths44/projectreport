\chapter{Conclusion}
The ASTRO platform is a comprehensive solution designed to empower small-scale vendors by addressing their core operational challenges. By facilitating collaborative purchasing, optimizing logistics, and offering data-driven financial services, ASTRO significantly enhances the competitive edge of small vendors, enabling them to compete more effectively with large retail chains. Its multifaceted approach not only reduces operational costs and improves efficiency but also equips vendors with financial resources and data-driven insights essential for informed decision making and strategic growth.

Through collaborative purchasing, ASTRO enables vendors to access bulk pricing discounts, reducing the cost of goods sold and increasing profitability. The platform’s logistics optimization feature streamlines delivery routes, minimizing transportation costs and ensuring timely deliveries. This not only enhances operational reliability but also improves customer satisfaction. Additionally, ASTRO integrates data driven financial services to address the critical challenge of limited access to capital. By leveraging transaction data, the platform offers tailored credit assessments and loan options, empowering vendors to expand their businesses, manage inventory effectively, and invest in technology.

A key component of ASTRO is its real-time demand forecasting tool, powered by NeuralProphet. This predictive capability enables vendors to anticipate market demand with high accuracy, optimizing inventory levels and reducing the risks of overstocking or stockouts. By ensuring vendors can meet customer needs promptly and efficiently, ASTRO enhances overall operational efficiency. The platform is also designed with a user-friendly interface, making its features accessible to vendors with varying levels of technical expertise.

Overall, ASTRO fosters economic resilience and sustainability within local communities by empowering small-scale vendors to thrive in a competitive marketplace. By addressing critical challenges such as limited purchasing power, inefficient logistics, restricted financial access, and a lack of data driven decision making, ASTRO contributes to the long term success and stability of small vendors. In doing so, it promotes a more diverse and robust local economy.

Our implementation demonstrates significant improvements in supply chain efficiency and cost reduction through order aggregation and route optimization. The advanced forecasting capabilities powered by NeuralProphet provide vendors with unprecedented insights into future demand patterns, as evidenced by our comprehensive analysis results.

To evaluate the effectiveness of the proposed collaborative ordering and shared delivery model, we analyzed route optimization performance across 50 vendors, comparing traditional individual delivery routes against our collaborative approach. In the 50-vendor scenario presented in Table~\ref{tab:cost-comparison}, our system processed five staggered orders of varying sizes, analyzing both traditional and collaborative routing methods. The traditional approach required five separate delivery routes, often with significant overlap. In contrast, our collaborative model consolidated these into a single optimized route, yielding substantial savings in distance, cost, and fuel consumption.
\begin{itemize}
    \item {Scenario:} 50 vendors requiring deliveries from a central supplier
    \item {Traditional approach:} Individual routes planned for each vendor or small group
    \item {Collaborative approach:} Single optimized route serving all vendors
    \item {Distance analyzed:} Regional delivery network covering approximately 7,000 km
\end{itemize}

\subsubsection{Cost Comparison Analysis}

\begin{table}[htbp]
    \centering
    \caption{Cost Comparison: Traditional vs. Collaborative Route Optimization}
    \label{tab:cost-comparison}
    \begin{tabular}{lrrr}
        \toprule
        \textbf{Metric}  & \textbf{Trad. Approach} & \textbf{Collab. Approach} & \textbf{Savings} \\
        \midrule
        Distance (km)    & 35,606                  & 6,993                     & 28,613 (80.4\%)  \\
        Route Cost (Rs.) & 36,929                  & 7,255                     & 29,675 (80.4\%)  \\
        Fuel (L)         & 1,811                   & 357                       & 1,454 (80.3\%)   \\
        Fuel Cost (Rs.)  & 171,135                 & 33,770                    & 137,366 (80.3\%) \\
        \bottomrule
    \end{tabular}
\end{table}

\paragraph{Traditional Delivery Approach} In the conventional model, vendors receive deliveries through separate routes or small grouped deliveries, resulting in significant route overlap and inefficiency. This approach required a total travel distance of 35,606.06 km with associated route costs of Rs. 36,929.20. The excessive distance resulted in high fuel consumption (1,811.34 L) and fuel costs (Rs. 171,135.40).

\paragraph{Collaborative Delivery Approach} With our platform, deliveries are consolidated into a single optimized route serving all vendors. Our Environmental-Adaptive Genetic Algorithm (EAGA) produced a route of just 6,993.30 km with total costs of Rs. 7,254.56. This dramatically reduced fuel consumption to 357.43 L with corresponding fuel costs of Rs. 33,769.57.

\paragraph{Impact Assessment} As shown in Table~\ref{tab:cost-comparison}, the collaborative delivery model demonstrates an 80.4\% reduction in total distance traveled and route costs, with fuel savings of 80.3\%. These substantial efficiencies are achieved through intelligent route optimization that eliminates redundant travel paths. The model provides small-scale vendors with logistics cost advantages traditionally reserved for large enterprises, enabling them to compete more effectively while maintaining their operational independence.

\par The results of our implementation validate the core thesis that technology-driven collaboration can substantially improve the competitive position of small-scale vendors in today's market. By combining advanced machine learning techniques with practical business solutions, ASTRO demonstrates that sophisticated technologies can be made accessible and valuable to small businesses that traditionally lack access to such tools. The platform's ability to decompose complex demand patterns into interpretable components—trend, weekly cycles, and annual seasonality—empowers vendors with actionable intelligence previously available only to large retail corporations with dedicated analytics teams.

\begin{figure}[htbp]
    \centering
    \begin{tikzpicture}
        \begin{axis}[
                width=\columnwidth,
                height=6cm,
                xlabel={Time Period},
                ylabel={Order Quantity},
                legend pos=north west,
                grid=both,
                date coordinates in=x,
                xticklabel={\year},
                xticklabel style={rotate=45, anchor=north east},
                xtick distance=365,
                date ZERO=2023-01-01,  % Reference start date
                ymajorgrids=true,
                xmajorgrids=true,
                legend style={font=\footnotesize},
                legend cell align={left},
            ]

            % Actual data from training dataset
            \addplot[
                blue,
                thick,
                smooth
            ]
            table[
                    x=ds,
                    y=y,
                    col sep=comma
                ] {Figures/np_train_data.csv};
            \addlegendentry{Orders}

            % Forecasted data
            \addplot[
                red,
                thick,
                smooth
            ]
            table[
                    x=ds,
                    y=yhat1,
                    col sep=comma
                ] {Figures/forecast.csv};
            \addlegendentry{Forecast}
        \end{axis}
    \end{tikzpicture}
    \caption{NeuralProphet Demand Forecasting}
    \label{fig:forecast}
\end{figure}
\begin{figure}[htbp]
    \centering
    \begin{tikzpicture}
        \begin{axis}[
                width=\columnwidth,
                height=6cm,
                xlabel={Year},
                ylabel={Order Quantity},
                legend pos=north west,
                grid=both,
                date coordinates in=x,
                xtick={2018-01-01, 2019-01-01, 2020-01-01, 2021-01-01, 2022-01-01, 2023-01-01, 2024-01-01,2025-01-01},
                xticklabels={2018, 2019, 2020, 2021, 2022, 2023, 2024,2025},
                xticklabel style={rotate=45, anchor=north east},
                date ZERO=2018-01-01,
                ymajorgrids=true,
                xmajorgrids=true,
                legend style={font=\footnotesize},
                legend cell align={left},
            ]

            % Actual data from training dataset
            \addplot[
                blue,
                thick,
            ]
            table[
                    x=ds,
                    y=trend,
                    col sep=comma
                ] {Figures/forecast_whole.csv};
            \addplot[
                red,
                thick,
            ]
            table[
                    x=ds,
                    y=trend,
                    col sep=comma
                ] {Figures/forecast.csv};
        \end{axis}
    \end{tikzpicture}
    \caption{NeuralProphet Long-term Trend Analysis}
    \label{fig:trend}
\end{figure}
\begin{figure}[H]
    \centering
    \begin{tikzpicture}
        \begin{axis}[
                width=\columnwidth,
                height=6cm,
                xlabel={Day of Week},
                ylabel={Seasonal Effect},
                legend pos=north west,
                grid=both,
                ytick={-30,-20,-10,0,10,20,30},              % Specify exact tick positions
                yticklabels={-30,-20,-10,0,10,20,30},
                xtick={0,1,2,3,4,5,6,7},
                xticklabels={Sunday,Monday,Tuesday,Wednesday,Thursday,Friday,Saturday,Sunday},
                xticklabel style={rotate=45, anchor=north east},
                ymajorgrids=true,
                xmajorgrids=true,
                legend style={font=\footnotesize},
                legend cell align={left},
            ]

            % Actual data from training dataset - only first 7 entries with smooth curve
            \addplot[
                blue,
                thick,
                smooth,
                tension=0.4, % Add some tension for a smoother curve
            ]
            table[
                    x expr=\coordindex, % Use index as x-coordinate instead of date
                    y=season_weekly,
                    col sep=comma,
                    restrict expr to domain={\coordindex}{0:7}
                ] {Figures/forecast.csv};
        \end{axis}
    \end{tikzpicture}
    \caption{NeuralProphet Weekly Seasonality Analysis}
    \label{fig:weekly-seasonality}
\end{figure}
\begin{figure}[htbp]
    \centering
    \begin{tikzpicture}
        \begin{axis}[
                width=\columnwidth,
                height=6cm,
                xlabel={Month},
                ylabel={Seasonal Effect},
                legend pos=north west,
                grid=both,
                date coordinates in=x,
                ytick={-60,-40,-20,0,20,40,60},              % Specify exact tick positions
                yticklabels={-60,-40,-20,0,20,40,60},
                xticklabel={\month}, % Show just month number
                xtick distance=91.5,  % Approximately monthly ticks
                date ZERO=2023-01-01,
                xtick={2023-01-01,2023-03-01,2023-05-01,2023-07-01,2023-09-01,2023-11-01,2024-01-01}, % Explicit tick positions
                xticklabels={Jan,Mar,May,Jul,Sep,Nov,Jan}, % Month labels
                xticklabel style={rotate=45, anchor=north east},
                ymajorgrids=true,
                xmajorgrids=true,
                legend style={font=\footnotesize},
                legend cell align={left},
            ]

            % Actual data from training dataset
            \addplot[
                blue,
                thick,
                smooth,
                tension=0.4
            ]
            table[
                    x=ds,
                    y=season_yearly,
                    col sep=comma
                ] {Figures/forecast.csv};
        \end{axis}
    \end{tikzpicture}
    \caption{NeuralProphet Yearly Seasonality Analysis}
    \label{fig:yearly-seasonality}
\end{figure}
\par Fig.~\ref{fig:forecast} illustrates NeuralProphet's ability to predict future order quantities based on historical data, providing vendors with reliable projections for inventory planning. Supporting this predictive capability, Fig.~\ref{fig:trend} reveals the underlying long-term growth trajectory that informs strategic business decisions and expansion planning. Further enhancing these insights, Fig.~\ref{fig:weekly-seasonality} and Fig.~\ref{fig:yearly-seasonality} demonstrate how our model accurately captures both weekly variations and annual seasonal patterns in vendor ordering behavior. This comprehensive forecasting capability is crucial for vendors and suppliers to anticipate demand fluctuations, optimize inventory management, and plan logistics operations efficiently throughout the year.

\par The results of our implementation validate the core thesis that technology-driven collaboration can substantially improve the competitive position of small-scale vendors in today's market. By combining advanced machine learning techniques with practical business solutions, ASTRO demonstrates that sophisticated technologies can be made accessible and valuable to small businesses that traditionally lack access to such tools. The platform's ability to decompose complex demand patterns into interpretable components—trend, weekly cycles, and annual seasonality empowers vendors with actionable intelligence previously available only to large retail corporations with dedicated analytics teams.
