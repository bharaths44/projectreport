\chapter{Literature Survey}

X. Wang and L. Wang, in their study titled \textit{"Green Routing Optimization for Logistics Distribution with Path Flexibility and Service Time Window"}, presented at the 2021 15th International Conference on Service Systems and Service Management (ICSSSM) in Hangzhou, China, address optimizing logistics routes by focusing on minimizing environmental impact (especially CO\textsubscript{2} emissions) and allowing path flexibility based on real-time conditions. The study also considers service time windows to ensure timely deliveries.

\textbf{Information and Methods}

The study investigates two main factors: green routing, which minimizes environmental impact, and path flexibility, which allows choosing alternate routes based on real-time conditions. Additionally, service time windows are incorporated to ensure deliveries are made within specified time frames.

\textbf{Optimization Model}

The authors developed a multi-objective optimization model that considers:
\begin{itemize}
    \item \textbf{Green routing objectives}: Minimizing CO\textsubscript{2} emissions.
    \item \textbf{Logistical objectives}: Minimizing delivery time, distance, and transportation costs.
    \item \textbf{Service time windows}: Ensuring deliveries happen within pre-set time intervals.
\end{itemize}

An improved genetic algorithm (GA) is proposed to solve this optimization problem. The algorithm uses evolutionary principles (selection, crossover, and mutation) to iteratively generate and evolve solutions.

\textbf{Model Inputs}
\begin{itemize}
    \item \textbf{Delivery points}: Locations for each delivery.
    \item \textbf{Time windows}: Constraints on the time within which deliveries must occur.
    \item \textbf{Vehicle capacities}: The size and limitations of vehicles.
    \item \textbf{Road network data}: Information on available routes and traffic conditions.
\end{itemize}

\textbf{Data Preparation and Datasets}

Data was sourced from real-world logistics companies and road network databases, containing information on delivery points, road conditions, traffic patterns, and environmental factors. Data preprocessing includes handling missing values, outliers, and incorrect time windows.

\textbf{Results}

The genetic algorithm generated optimized routes that minimized both CO\textsubscript{2} emissions and total delivery time, while allowing flexibility in route selection. The model also met service time windows, ensuring customer satisfaction.

\textbf{Advantages}
\begin{enumerate}
    \item \textbf{Environmental Impact Reduction}: Green routing reduces carbon emissions, contributing to sustainable logistics operations.
    \item \textbf{Flexibility in Routing}: Path flexibility allows adaptation to real-time traffic conditions, road closures, and weather disruptions.
    \item \textbf{Timeliness of Delivery}: Meeting service time windows improves operational efficiency and customer satisfaction.
    \item \textbf{Multi-Objective Optimization}: Optimizes cost, time, and environmental impact, providing a comprehensive solution for logistics.
\end{enumerate}

\textbf{Disadvantages}
\begin{enumerate}
    \item \textbf{Computational Complexity}: The genetic algorithm can be computationally intensive, especially for large networks.
    \item \textbf{Dependency on Traffic and Road Data}: Optimization accuracy depends on reliable traffic and road data.
    \item \textbf{Limited Scalability}: May struggle to scale for large logistics systems with many variables.
\end{enumerate}

A. Maroof, B. Ayvaz, and K. Naeem, in their study titled \textit{"Logistics Optimization Using Hybrid Genetic Algorithm (HGA): A Solution to the Vehicle Routing Problem With Time Windows (VRPTW)"} published in IEEE Access, vol. 12, pp. 100245-100255, 2024, address the Vehicle Routing Problem with Time Windows (VRPTW). This research minimizes transportation costs while respecting delivery time windows.

\textbf{Hybrid Genetic Algorithm (HGA)}

The authors propose a Hybrid Genetic Algorithm (HGA) that combines a traditional genetic algorithm with local search methods like 2-opt and swap heuristics to improve solution quality and convergence speed, helping avoid local optima and ensuring a global solution.

\textbf{Optimization Goal}

The goal is to find optimal routes for a fleet of vehicles that minimizes total travel distance while meeting the time constraints of each delivery point.

\textbf{Algorithm Components}
\begin{itemize}
    \item \textbf{Genetic Algorithm (GA)}: Selection, crossover, and mutation to explore and exploit the solution space.
    \item \textbf{Local Search (2-opt and Swap)}: Post-processing of GA solutions to optimize routes and remove inefficiencies.
\end{itemize}

\textbf{Data Preparation and Datasets}

Similar to Wang and Wang’s study, the dataset includes vehicle capacities, service time windows, and road network data. Preprocessing ensures reasonable time windows and correct vehicle capacities.

\textbf{Results}

The HGA minimized total travel distance while meeting time windows and outperformed traditional genetic algorithms in terms of convergence speed and solution quality.

\textbf{Advantages}
\begin{enumerate}
    \item \textbf{Improved Solution Quality}: Hybridization avoids local optima and improves solution quality.
    \item \textbf{Handling Complex Constraints}: Effectively manages constraints like time windows and vehicle capacities.
    \item \textbf{Computational Efficiency}: More efficient than basic GAs, suitable for large-scale problems.
    \item \textbf{Adaptability}: Flexible for various VRPTW problems, including multiple depots and fleet sizes.
\end{enumerate}

\textbf{Disadvantages}
\begin{enumerate}
    \item \textbf{Local Optima Issues}: Despite improvements, careful tuning is required to avoid local optima.
    \item \textbf{Complexity in Hybridization}: Balancing GA and local search adds complexity.
    \item \textbf{Static Approach}: Does not account for real-time dynamic changes like traffic disruptions or sudden demand shifts.
\end{enumerate}

\textbf{Comparison and Integration}

\begin{itemize}
    \item \textbf{Green Routing} (Wang \& Wang, 2021): Emphasizes sustainability through green routing, reducing environmental impact, a valuable feature for eco-conscious platforms.
    \item \textbf{Hybrid Genetic Algorithm} (Maroof et al., 2024): Ensures time-sensitive, cost-efficient deliveries, ideal for small-scale vendors requiring reliable logistics.
\end{itemize}

\textbf{Conclusion}

Combining green routing (Wang \& Wang, 2021) and optimized delivery scheduling (Maroof et al., 2024) provides environmental sustainability and logistical efficiency, making it highly suitable for optimizing small-scale vendor logistics.
Contents \cite{xu2023vitpose++}




