\chapter{Literature Survey}

\textbf{\cite{taylor2017forecasting}Taylor S. J., Letham B. , "Forecasting at Scale". PeerJ 5 Preprints: e3190v2,2017}

The authors  present a flexible and robust forecasting tool designed to handle complex business time-series data with ease and accuracy. Developed primarily for non-experts, Prophet aims to overcome the limitations of traditional forecasting models like ARIMA and exponential smoothing, which often require extensive domain knowledge and struggle with irregular data patterns. The motivation behind Prophet is to create a scalable and intuitive model that can adapt to diverse business forecasting needs, such as demand prediction, financial planning, and inventory management.

The methodology behind Prophet is based on a decomposable time-series model comprising three core components: Trend, Seasonality, and Holiday Effects. The trend is modeled using either a piecewise linear or saturating growth function, allowing it to adapt to various growth behaviors. Seasonality is captured using Fourier series, which can handle multiple seasonal patterns effectively. Holiday effects account for custom, periodic events that impact the time series. Additionally, Prophet includes Automatic Changepoint Detection, which identifies shifts in trends using a sparse prior to ensure the model adjusts flexibly to changes in the data.

The model is highly interactive, allowing analysts to incorporate domain knowledge by tweaking trend changepoints, seasonality, and holiday components. This "analyst-in-the-loop" capability ensures that forecasts are not only data-driven but also aligned with business insights. Furthermore, Prophet’s scalability enables it to handle large volumes of data, making it suitable for organizations with multiple time-series datasets.

Prophet was evaluated against traditional forecasting methods and demonstrated superior performance, especially in scenarios with complex seasonal patterns and irregularities. Its ability to handle missing data, outliers, and varying frequencies contributed to its robustness. The model is also user-friendly, with parameters that are intuitive and easy to configure, making it accessible to both data scientists and business analysts without deep statistical expertise.

One of the significant contributions of this paper is its emphasis on flexibility and interpretability in forecasting. By providing a model that is both accurate and customizable, Prophet bridges the gap between sophisticated statistical tools and practical business applications. However, the authors acknowledge limitations such as the model's dependency on well-defined seasonalities and its difficulty in capturing highly non-linear patterns without additional input from the analyst.

The paper concludes by emphasizing Prophet’s practical utility in real-world business environments. Future research directions include enhancing the model to handle multi-variate time-series data and integrating it with machine learning techniques for greater predictive power. Overall, Prophet represents a significant advancement in time-series forecasting, providing a powerful yet accessible tool for data-driven decision-making.

\textbf{\cite{wang2018greenrouting}X. Wang and L. Wang, "Green Routing Optimization for Logistics Distribution with Path Flexibility and Service Time Window," 2018 15th International Conference on Service Systems and Service Management (ICSSSM), Hangzhou, China, 2018, pp. 1-5, doi: 10.1109/ICSSSM.2018.8464993.}

The authors discuss the implementation of a time-dependent vehicle routing problem to optimize green logistics. The study emphasizes reducing fuel consumption and CO2 emissions by developing an algorithm that incorporates dynamic traffic conditions and customer service time windows. Unlike traditional routing models, the authors integrate real-world factors such as vehicle speed variations based on traffic, road conditions, and time of day, enabling a more practical application in urban logistics scenarios. The motivation for the research lies in enhancing sustainability by minimizing the carbon footprint of logistics operations while improving cost efficiency and customer satisfaction.

The methodology involves formulating an optimization model that minimizes both CO2 emissions and travel time. The routing problem includes constraints like vehicle capacity, service time windows, and varying traffic conditions. A unique feature of the model is its objective function, which calculates emissions based on vehicle speed, distance, and load. This approach incorporates a non-linear emission rate function that reflects the impact of speed variations on fuel consumption and emissions. Another innovative aspect is the flexibility in path selection, allowing vehicles to choose optimal routes based on real-time traffic data.

To validate the model, the authors tested it on real-world logistics scenarios, comparing its performance against conventional methods. The results demonstrate significant reductions in both CO2 emissions and travel time, showcasing the effectiveness of considering dynamic traffic conditions. The model proved especially efficient in urban environments with high traffic variability, offering valuable insights for logistics managers seeking to balance cost savings with environmental responsibility.

One of the key contributions of this paper is its focus on sustainability in logistics. By addressing the dual objectives of environmental impact and operational efficiency, the study highlights the importance of adopting green logistics strategies in modern supply chains. The use of time-dependent speeds and flexible path selection represents a substantial improvement over static models, making the approach highly relevant for real-world applications. However, the authors also acknowledge limitations, such as the complexity of implementing such models at scale and the potential variability in traffic data accuracy.

The paper concludes by emphasizing the practical benefits of integrating environmental considerations into logistics planning. The proposed model not only achieves cost savings and emissions reductions but also aligns with the growing demand for sustainable business practices. Future research directions include extending the model to account for multi-modal transportation systems and exploring the integration of renewable energy-powered vehicles into the routing framework. Overall, this study provides a comprehensive solution for green logistics optimization, contributing significantly to the field of sustainable supply chain management.

\textbf{\cite{ekanayake2020antcolony}E. M. U. S. B. Ekanayake, S. P. C. Perera, W. B. Daundasekara, and Z. A. M. S. Juman, "A Modified Ant Colony Optimization Algorithm for Solving a Transportation Problem," Journal of Advances in Mathematics and Computer Science, vol. 35, no. 5, pp. 83-101, 2020}

The authors present a novel approach to addressing logistics transportation issues using an enhanced version of the Ant Colony Optimization (ACO) algorithm. The study focuses on minimizing transportation costs for both balanced and unbalanced problems while improving computational efficiency. Unlike traditional methods such as Vogel’s Approximation Method (VAM), the proposed Modified Ant Colony Optimization Algorithm (MACOA) adapts pheromone updates and transition rules to provide superior solutions, particularly for large-scale logistics problems. The motivation behind the research stems from the increasing demand for cost-effective and time-efficient solutions in transportation logistics.

The methodology begins with converting unbalanced transportation problems into balanced ones by introducing dummy sources or destinations. Ants, which symbolize agents in the ACO framework, calculate probabilities based on a transition rule to allocate resources efficiently. The ants iteratively construct solutions by selecting routes with the highest probabilities, ensuring all demands and supplies are met. The pheromone update mechanism is customized to strengthen promising routes while discouraging suboptimal ones. By incorporating adjustments to the initialization and update phases, the algorithm enhances the traditional ACO approach, achieving faster convergence and improved solution quality.

The authors tested MACOA against benchmark transportation problems, comparing its performance to VAM and other established methods. Results indicate that MACOA consistently outperformed VAM in minimizing transportation costs while achieving comparable results to more computationally intensive techniques. Additionally, the algorithm demonstrated significant improvements in solving large-scale problems, offering time savings without compromising accuracy. The study highlights the algorithm's potential in real-world applications, particularly in industries where efficient logistics and supply chain management are critical.

The key contributions of this work include its ability to handle unbalanced transportation problems effectively and its computational efficiency for large datasets. By reducing the number of iterations needed to achieve optimal or near-optimal solutions, MACOA provides a practical alternative to traditional methods. However, the authors acknowledge that the algorithm, being heuristic, does not guarantee exact optimal solutions in all scenarios. They suggest future work to refine the method, aiming to balance efficiency with precision and scalability.

The study concludes that MACOA represents a significant advancement in transportation optimization, particularly for large-scale problems where computational resources are constrained. The proposed algorithm offers a blend of cost reduction and computational speed, making it an attractive option for logistics managers. Future research could explore integrating MACOA with hybrid optimization techniques or adapting it to dynamic transportation scenarios involving real-time changes in supply and demand. Overall, the paper demonstrates the potential of modified ACO algorithms in addressing complex logistics challenges, contributing valuable insights to the field of transportation optimization.

\textbf{\cite{ai2021multisupplier}KAi, X., Yue, Y., Xu, H., and Deng, X. "Optimizing multi-supplier multi-item joint replenishment problem for non-instantaneous deteriorating items with quantity discounts," PLoS ONE, 16(2), e0246035, 2021}

The authors address the challenges of managing inventory in supply chains dealing with perishable goods and multiple suppliers. The study focuses on creating a joint replenishment strategy to minimize costs while considering the perishability of items and supplier-specific quantity discounts. The motivation for this research arises from the inefficiencies faced by retailers when ordering independently, leading to higher costs and suboptimal inventory management. The proposed model seeks to balance replenishment timing and order quantities to reduce overall supply chain costs.

The methodology employs a Mixed-Integer Nonlinear Programming (MINLP) model to optimize total costs, including ordering, holding, and deterioration costs. The authors consider factors such as demand rate, preservation period, and deterioration rate for each item, along with supplier-specific discount schemes. To solve this complex optimization problem, the study uses an Improved Moth-Flame Optimization (IMFO) algorithm. This metaheuristic method enhances traditional optimization by avoiding local optima through techniques like hyperbolic spiral functions and Levy-flight strategies. The IMFO is designed to achieve faster convergence and higher accuracy compared to other algorithms such as Genetic Algorithm (GA), Particle Swarm Optimization (PSO), and Grey Wolf Optimizer (GWO).

The results highlight the effectiveness of the IMFO in solving joint replenishment problems. The algorithm demonstrated cost savings of 0.01\% to 1.35\% compared to traditional optimization methods, along with faster convergence rates. Sensitivity analysis revealed that demand rates significantly influence total costs, while supplier selection remained consistent across varying parameters. The model effectively optimized replenishment schedules for deteriorating items, ensuring cost-efficiency and better inventory turnover.

A key contribution of the study is its ability to handle multi-supplier and multi-item scenarios with non-instantaneous deteriorating goods. By incorporating supplier-specific quantity discounts, the model provides a practical solution for retailers aiming to leverage economies of scale. Additionally, the use of IMFO ensures computational efficiency, making the approach suitable for large-scale problems. However, the authors acknowledge certain limitations, including the reliance on static demand rates and the challenges of applying the model to dynamic environments with real-time data.

The paper concludes by emphasizing the applicability of the proposed methodology to supply chains dealing with perishable goods. Future research could focus on integrating real-time data and dynamic demand forecasting into the optimization framework or exploring hybrid approaches to further enhance solution accuracy. Overall, the study provides valuable insights into joint replenishment strategies, contributing to the optimization of inventory management in complex supply chain systems.

\textbf{\cite{triebe2021neuralprophet}O. Triebe, H. Hewamalage, P. Pilyugina, N. Laptev, C. Bergmeir, and R. Rajagopal, "NeuralProphet: Explainable Forecasting at Scale" , arXiv preprint, Nov. 2021}


The authors introduce an enhanced time-series forecasting model that builds upon Facebook’s Prophet framework. NeuralProphet integrates neural networks and traditional statistical methods to improve accuracy, scalability, and explainability. The motivation behind this research is to address the limitations of traditional models like ARIMA and Prophet, which often struggle with capturing local dynamics, non-linear trends, and interactions in complex time-series data. NeuralProphet aims to bridge the gap by offering a more flexible, robust, and interpretable solution suitable for large-scale forecasting tasks.

The methodology centers on a modular model structure that decomposes time-series data into components, such as trend, seasonality, event effects, and auto-regression. Each component is modeled independently using both classical techniques and neural network-based extensions. The model’s key innovation lies in its hybrid nature, combining additive and multiplicative effects for better flexibility. Features like auto-regression and lagged regressors enable the model to capture local patterns and dependencies in the data. The model also incorporates a deep-learning-powered changepoint detection mechanism to identify both abrupt and gradual shifts in trends effectively.

NeuralProphet was tested across diverse datasets and compared with other forecasting models, including ARIMA, Prophet, and modern neural network-based methods. The results showed significant improvements in accuracy, with NeuralProphet outperforming its predecessors by 55–92\% on various metrics. The model demonstrated robustness in handling high-frequency and multi-seasonal data while retaining scalability for large datasets. Its explainability features allowed for a detailed understanding of each component’s contribution to the forecast, enhancing its usability for analysts.

The study’s main contribution lies in providing a forecasting tool that balances flexibility, accuracy, and interpretability. By combining the strengths of traditional and neural network-based approaches, NeuralProphet offers a practical solution for complex time-series forecasting tasks. Its ability to decompose forecasts into understandable components makes it particularly valuable for non-experts and domain specialists. However, the authors acknowledge that the model’s performance depends on proper configuration and that further improvements in scalability and real-time forecasting capabilities are needed.

The paper concludes by emphasizing the potential of NeuralProphet in business and industry applications, particularly in scenarios requiring interpretable and scalable forecasting solutions. Future research could explore extending the model’s capabilities to multi-variate and hierarchical time-series data or integrating external factors such as economic indicators. Overall, the study highlights NeuralProphet as a significant advancement in time-series forecasting, offering a powerful and user-friendly alternative to existing models.

\textbf{\cite{siham2022genetic} Ben Jouida, S., and Krichen, S. (2020). A genetic algorithm for supplier selection problem under collaboration opportunities. Journal of Experimental and Theoretical Artificial Intelligence, 34(1), 53–79.}

The authors propose a framework to optimize order distribution and demand sharing across supply chain participants. The study addresses challenges such as limited information sharing, unequal benefit distribution, and inefficient coordination in collaborative supply chains. By leveraging a heuristic approach based on multi-criteria optimization, the authors aim to enhance collaboration, reduce inefficiencies, and achieve equitable decision-making among supply chain stakeholders. The motivation stems from the need to support demand-driven environments where real-time coordination is essential for supply chain effectiveness.

The methodology revolves around a central coordination system designed to manage demand allocation and order distribution across a three-tier supply chain. The framework integrates the Analytic Hierarchy Process (AHP) to prioritize criteria influencing demand allocation and employs Genetic Algorithms (GA) for iterative optimization of order distribution. Collaboration rules are predefined to ensure equitable demand sharing and fairness among participants. The GA evaluates solutions based on criteria like cost efficiency, resource utilization, and adherence to the collaboration rules, refining them through evolutionary operations such as selection, crossover, and mutation.

The results demonstrate the framework's effectiveness in improving supply chain coordination and achieving balanced demand allocation. Using the heuristic approach, the system successfully optimized order distribution, leading to significant cost reductions and enhanced operational efficiency. The methodology’s scalability was validated in real-world scenarios, proving its adaptability to various supply chain structures and demands. Additionally, the integration of AHP ensured that critical factors, such as fairness and resource prioritization, were consistently considered in the optimization process.

A key contribution of the paper is its demonstration of how heuristic methods can overcome the complexities of demand-driven collaborative supply chains. By emphasizing equitable benefit distribution and operational efficiency, the proposed framework fosters sustained cooperation among supply chain participants. However, the authors acknowledge certain limitations, such as the dependence on predefined collaboration rules and the computational demands of genetic algorithms for large-scale implementations.

The study concludes by highlighting the potential of heuristic methodologies to transform collaborative supply chains. The proposed framework offers a practical and scalable solution for real-world applications, particularly in environments requiring high levels of coordination and fairness. Future research directions include exploring the integration of real-time data analytics into the framework and expanding the model to accommodate dynamic changes in supply chain demands. Overall, the paper provides valuable insights into optimizing collaborative supply chains, contributing significantly to the field of demand-driven supply chain management.


\textbf{\cite{seifbarghy2022coordination}LSeifbarghy, M., Shoeib, M., and Pishva, D. "Coordination of a single-supplier multi-retailer supply chain via joint ordering policy considering incentives for retailers and utilizing economies of scale," International Journal of Production Economics, 151, 49–59, 2022}

The authors of present a collaborative framework to optimize inventory management and reduce costs in supply chains involving small retailers. The research addresses challenges such as high individual ordering costs, inefficiencies in inventory management, and missed opportunities for economies of scale. By introducing a joint ordering policy, the study aims to enhance coordination between retailers and suppliers, leveraging bulk purchasing benefits while maintaining equitable cost-sharing.

The methodology focuses on developing a Mathematical Optimization Model to determine optimal joint ordering schedules and quantities. The model incorporates Mixed-Integer Linear Programming (MILP) to minimize total costs, including ordering, holding, and shortage costs, while considering constraints like demand rates, lead times, and supplier capacities. A central feature of the framework is its Incentive Mechanism, which distributes cost savings among retailers based on their contribution to the joint order. This encourages collaboration by ensuring fair benefit-sharing. The model also accounts for economies of scale by synchronizing ordering periods, enabling bulk discounts from suppliers and reducing logistics costs.

The results demonstrate that the joint ordering policy significantly reduces overall supply chain costs. Retailers benefit from lower procurement expenses due to volume discounts, while suppliers gain from streamlined production and delivery schedules. The proposed incentive mechanism effectively fosters collaboration, ensuring sustained participation from all retailers. Furthermore, the model improves inventory turnover rates and minimizes stockouts, enhancing supply chain efficiency and responsiveness.

One of the key contributions of this paper is its practical applicability for small and medium-sized enterprises (SMEs). The framework provides a scalable solution that enables smaller retailers to compete with larger counterparts by pooling resources and optimizing procurement strategies. However, the study notes limitations such as the reliance on static demand forecasts and the need for robust information sharing among participants for successful implementation.

The authors conclude by emphasizing the importance of collaborative strategies in modern supply chains. Future research could focus on incorporating dynamic demand forecasting and real-time data analytics to enhance the model's adaptability. Additionally, exploring applications in multi-tier supply chains or integrating sustainability metrics could further expand its utility. Overall, the study offers a significant contribution to supply chain optimization, providing a roadmap for achieving cost-efficiency and collaboration in single-supplier, multi-retailer systems.

\textbf{\cite{zhou2023whalealgorithm}J. Zhou, "Research on Multi-objective  Vehicle Path Optimization Based on Whale Algorithm," 2023 International Conference on Networking, Informatics and Computing (ICNETIC), Palermo, Italy, 2023}

The authors of “Multi-objective Vehicle Path Optimization Based on Whale Algorithm” explore the use of a Modified Whale Optimization Algorithm (WOA) to address vehicle routing challenges by balancing cost, delivery efficiency, and vendor collaboration. This research focuses on optimizing logistics operations by improving delivery efficiency and ensuring coordination among vendors. The motivation behind this study stems from the growing demand for efficient logistics solutions that minimize operational costs while addressing the complexities of multi-vendor systems.

The methodology leverages the biological behavior of whales, particularly their hunting strategies, to solve multi-objective optimization problems. The algorithm employs two primary strategies: the Encircling Prey Strategy, which adjusts the whale’s position based on proximity to the best-known solution, and the Spiral Movement Strategy, which simulates a helical path around the optimal solution. These strategies enable the algorithm to balance exploration (searching new solutions) and exploitation (refining current solutions). The authors further adapt the WOA for multi-objective scenarios by introducing parameters that optimize delivery cost and efficiency simultaneously.

The algorithm's performance was evaluated using standard benchmarks and compared with single-objective optimization methods. Results showed that the Modified WOA achieved superior outcomes, demonstrating reduced operational costs and faster convergence to optimal solutions. The multi-objective approach allowed the algorithm to account for trade-offs between competing goals, such as minimizing costs while ensuring timely deliveries. Additionally, the adaptive mechanisms of WOA improved its ability to handle diverse logistics scenarios, including those with dynamic constraints like fluctuating customer demands and delivery time windows.

One of the key contributions of the paper is the demonstration of WOA’s applicability to real-world logistics challenges. Unlike traditional optimization techniques, which often focus on single objectives, this algorithm provides a comprehensive solution that addresses multiple logistics goals simultaneously. The study also highlights the algorithm’s scalability and computational efficiency, making it suitable for large-scale problems in supply chain management. However, the authors note that while WOA effectively balances multiple objectives, its performance may be influenced by parameter tuning, necessitating further exploration to improve robustness.

The paper concludes by emphasizing the practical benefits of using Modified WOA for vehicle routing problems. The algorithm’s ability to optimize cost, efficiency, and collaboration offers a competitive edge in logistics management. Future work could focus on integrating WOA with other optimization methods, such as hybrid algorithms, or applying it to emerging areas like drone-based logistics and autonomous vehicle routing. Overall, the study showcases the potential of bio-inspired algorithms in solving complex multi-objective problems, contributing to advancements in efficient and sustainable logistics operations.

\textbf{\cite{mamoun2024pharmaceutical}Mamoun, K.A., Hammadi, L., Ballouti, A.E., Novaes, A.G.N.. and Cursi, E.S.D. Vehicle Routing Optimization Algorithms for Pharmaceutical Supply Chain: A Systematic Comparison. Transport and Telecommunication Journal, Sciendo, Vol. 25 (Issue 2), (2024) pp. 161-173.}

The authors of “Vehicle Routing Optimization Algorithms for Pharmaceutical Supply Chain: A Systematic Comparison” conduct a comprehensive evaluation of multiple algorithms to address the Vehicle Routing Problem (VRP) within the pharmaceutical supply chain. The focus of the study lies in comparing the efficiency and scalability of various optimization methods, including Branch and Cut, Clarke and Wright's Savings Algorithm, Tabu Search, and Simulated Annealing, to minimize delivery costs and improve logistical performance. The motivation stems from the critical nature of pharmaceutical supply chains, where timely and cost-effective deliveries are essential for ensuring consistent medication availability.

The methodology involves analyzing the input parameters of VRP, such as customer locations, delivery demands, vehicle capacity, and distance matrices, to evaluate the performance of the selected algorithms. Branch and Cut is employed as an exact algorithm to generate optimal solutions for smaller problems. Meanwhile, heuristic methods like Clarke and Wright's Savings Algorithm and metaheuristic approaches such as Tabu Search and Simulated Annealing are applied to solve larger and more complex problems. The study assesses the algorithms' performance based on metrics like computational time, cost minimization, and solution accuracy.

The results highlight the strengths and weaknesses of the various algorithms. Branch and Cut delivers optimal solutions with a minimal gap percentage but struggles with scalability due to its computational intensity. Clarke and Wright's heuristic is fast and simple but less effective for large-scale problems. In contrast, metaheuristic methods, such as Tabu Search and Simulated Annealing, efficiently handle complex scenarios, achieving near-optimal solutions while maintaining computational feasibility. The study demonstrates that Tabu Search, in particular, provides a balanced trade-off between solution quality and scalability, making it well-suited for large-scale pharmaceutical supply chains.

One of the key contributions of this paper is its systematic comparison of different algorithms, offering valuable insights into their applicability for specific scenarios within the pharmaceutical sector. The research underscores the importance of metaheuristic methods in addressing large and complex VRPs, where exact algorithms may fall short due to their computational demands. However, the study also acknowledges the limitations of metaheuristics, such as the need for parameter tuning and the potential for local optima.

The authors conclude by emphasizing the practical relevance of their findings. The choice of an algorithm should align with the specific requirements of the supply chain, such as problem size and the need for precision. Future research could explore hybrid approaches that combine the strengths of exact and heuristic methods or investigate real-time optimization techniques to address dynamic routing problems. Overall, this study makes a significant contribution to optimizing pharmaceutical supply chains by providing a detailed analysis of algorithmic performance, offering a guide for practitioners in selecting the most suitable routing solution.

\textbf{\cite{maroof2024hybridgenetic} A. Maroof, B. Ayvaz and K. Naeem, "Logistics Optimization Using Hybrid Genetic Algorithm (HGA): A Solution to the Vehicle Routing Problem With Time Windows (VRPTW)," IEEE Access, vol. 12, 2024}

The authors explore the use of an enhanced Genetic Algorithm (GA) to solve the Vehicle Routing Problem with Time Windows (VRPTW). This study introduces a hybrid approach combining GA with Solomon Insertion Heuristic (SIH) to optimize logistics operations by minimizing route distances and improving vehicle utilization. The motivation stems from the complexity of VRPTW, where traditional methods struggle to balance constraints like delivery time windows, cost-efficiency, and vehicle capacity. The authors aim to develop a solution that is computationally efficient while ensuring optimal logistics performance.

The methodology begins with population initialization, where a diverse set of high-quality initial routes is generated using SIH. This step ensures that the initial solutions are closer to optimal. The algorithm then iteratively refines the routes through GA operations, including selection, crossover, and mutation. Fitness evaluation is a key step where routes are assessed based on criteria such as travel distance, adherence to delivery time windows, and overall cost. The hybridization with SIH enhances the algorithm's capability to explore the solution space effectively, allowing for faster convergence to optimal or near-optimal solutions.

The performance of the Hybrid Genetic Algorithm was benchmarked against other algorithms, including CPLA and PITSH. Results demonstrated that HGA consistently outperformed its counterparts, achieving better route optimization and reduced computational time. The integration of SIH proved particularly effective in generating initial solutions that required fewer iterations to refine. Additionally, HGA showed scalability, maintaining its efficiency and accuracy even as the problem size increased.

One of the paper’s significant contributions is the development of a practical solution tailored for small and medium-sized logistics operations. By combining the exploratory power of GA with the problem-specific refinement capabilities of SIH, the authors present a method that is not only robust but also adaptable to various real-world scenarios. However, the study also notes certain limitations, such as the reliance on fine-tuning algorithm parameters and the potential for suboptimal solutions in highly dynamic environments.

The paper concludes by highlighting the potential of HGA as a versatile tool for solving complex logistics problems. The hybrid approach offers an effective balance between cost efficiency and computational feasibility, making it well-suited for small-scale vendors and localized supply chains. Future research could explore extending this approach to dynamic VRPTW scenarios or integrating it with other metaheuristic methods to improve adaptability. Overall, the study provides a substantial contribution to logistics optimization, demonstrating the effectiveness of hybrid algorithms in enhancing operational efficiency.
