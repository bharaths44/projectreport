\chapter{Introduction}
	
Chapter introduction goes here.
	\section{Background}
	
	This section should outline the background of the project under consideration highlighting current scenarios and its importance. Maximum content is around 1 page \cite{garg2019securing}.
	
	
	\section{Problem Definition}
	
	This section should mention the aim of the project. The problem should be defined in one or two sentences.
	

\begin{figure}[htbp]
		\centering
		\includegraphics[width=8.0cm]{Figures/car.jpeg}
		\caption{Insert your images here, and provide necessary captions.}
	\end{figure}
	
	
	\section{Scope and Motivation}
	
	This section mentions scope and motivation, which should be written as two paragraphs. The
	first paragraph describes the scope, whereas the second one describes the motivation. Write
	in around 5 sentences each. \\
	
	You may insert tables into your document using the given code:
	

	\begin{table}[!h]
		\begin{center}
		\caption{Insert table caption here}
			\begin{tabular}{|c|c|c|}
				\hline
				\textbf{Title 1} & \textbf{Title 2} & \textbf{Title 3} \\
				\hline
				1 & Content 1 & Content 2 \\
				\cline{1-2}
				2 & Content 3 & Content 4\\
				\hline
			\end{tabular}
		\end{center}
	\end{table}

	
	\section{Objectives}
	
	\begin{itemize}
		\item This section should be a numbered list. Five to six objectives are encouraged.
	\end{itemize}
	
	\section{Challenges}
	
	This section briefs the challenges involved in the project in two or three sentences.
	
	
	\section{Assumptions}
	
	This section briefs the assumptions in the project in two or three sentences or as a numbered list.
	
	
	\section{Societal / Industrial Relevance}
	
	This section describes where the project can be applied, either for the society or the industry. Write the relevance applicable for the work.
	
	\section{Organization of the Report}
	
	This section should outline a roadmap of the contents in the report.
	
\par 
Chapter conclusion goes here.