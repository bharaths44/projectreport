\chapter{Implementation}



The ASTRO platform has been implemented using a modern technology stack that ensures scalability, performance, and cross-platform compatibility. The implementation follows a client-server architecture with clear separation of concerns:

\begin{itemize}
    \item \textbf{Backend}: Developed using FastAPI, a modern, high-performance Python web framework
    \item \textbf{Frontend}: Built with Flutter to provide a seamless cross-platform experience for both mobile and web clients
    \item \textbf{Database}: Utilizes Firebase Firestore for flexible, scalable NoSQL document storage with real-time capabilities
    \item \textbf{Machine Learning Pipeline}: Implemented in Python using libraries such as NeuralProphet, scikit-learn, and pandas
\end{itemize}
\section{Backend Implementation}

\subsection{FastAPI Framework}

The backend system is implemented using FastAPI, a modern Python web framework designed for building high-performance APIs. FastAPI was chosen for several key advantages:

\begin{itemize}
    \item \textbf{Performance}: Built on Starlette and Pydantic, FastAPI offers near-native performance comparable to Node.js and Go
    \item \textbf{Type Checking}: Utilizes Python type hints for automatic validation and documentation
    \item \textbf{Asynchronous Support}: Native support for async/await patterns to handle concurrent requests efficiently
    \item \textbf{API Documentation}: Automatic generation of OpenAPI and Swagger UI documentation
    \item \textbf{Dependency Injection}: Built-in system for managing dependencies and services
\end{itemize}

\subsection{Database Schema}

The database schema is implemented using Firebase Firestore, a NoSQL document database that provides real-time synchronization and offline capabilities. The system uses Pydantic models for data validation and serialization:

Firestore's flexible document structure enables easy storage and retrieval of complex nested data, which is particularly beneficial for order items and shipping addresses. The system leverages Firestore's querying capabilities for efficient data filtering and real-time updates.
\subsection{API Endpoints}

The API is organized into several functional modules, each handling specific aspects of the platform:

Key API endpoints include:

\begin{enumerate}
    \item \textbf{Core Endpoints}:
          \begin{itemize}
              \item \texttt{GET /}: Root endpoint
              \item \texttt{GET /info}: System information
          \end{itemize}

    \item \textbf{Vendor Operations}:
          \begin{itemize}
              \item \texttt{GET /vendor/suppliers}: Get list of suppliers
              \item \texttt{GET /vendor/products/\{supplier\_id\}}: Get products from a supplier
              \item \texttt{POST /vendor/orders}: Place a new order
              \item \texttt{POST /vendor/instant\_order}: Place an order for immediate fulfillment
          \end{itemize}

    \item \textbf{Product Management}:
          \begin{itemize}
              \item \texttt{POST /supplier/\{supplier\_id\}/add\_product}: Add a new product
              \item \texttt{PUT /supplier/update\_product/\{product\_id\}}: Update product details
              \item \texttt{DELETE /supplier/delete\_product/\{product\_id\}}: Remove a product
              \item \texttt{GET /supplier/get\_products/\{supplier\_id\}}: Get supplier's products
          \end{itemize}

    \item \textbf{Order Aggregation}:
          \begin{itemize}
              \item \texttt{POST /supplier/aggregate\_orders}: Process orders for aggregation
              \item \texttt{POST /supplier/update\_instant\_order\_status}: Update status of instant orders
          \end{itemize}

    \item \textbf{Route Optimization}:
          \begin{itemize}
              \item \texttt{GET /route-optimization/\{aggregation\_id\}}: Generate optimal delivery route
          \end{itemize}

    \item \textbf{Demand Forecasting}:
          \begin{itemize}
              \item \texttt{GET /demand-forecast/\{product\_id\}}: Get demand forecast for a product
          \end{itemize}
\end{enumerate}

\section{Frontend Implementation}

\subsection{React and TypeScript Application}

The ASTRO platform frontend is implemented using React with TypeScript and Tailwind CSS, featuring a comprehensive routing structure and modern UI components. Our implementation features:

\begin{itemize}
    \item \textbf{Type-Safe Development}: TypeScript provides static typing throughout the codebase, enabling early error detection during development and improving code quality
    \item \textbf{Component Architecture}: A modular system of UI components organized by business domain (vendors, suppliers, orders) to maximize code reusability
    \item \textbf{Tailwind Integration}: Utility-first CSS approach through Tailwind enables rapid styling without context-switching between files
    \item \textbf{Responsive Layout}: Adaptive interface designs that work across desktop and mobile devices using Tailwind's responsive utility classes
    \item \textbf{Modern UI Components}: Custom component library including toasts, tooltips, and interactive elements for a consistent user experience
\end{itemize}

\subsection{Application Architecture}

Our React application follows a structured architecture organized around the following principles:

\begin{itemize}
    \item \textbf{React Query Integration}: Centralized data fetching through QueryClientProvider for efficient server state management
    \item \textbf{Context-Based Authentication}: Custom AuthProvider context manages user authentication state and permissions
    \item \textbf{Route-Based Code Structure}: Features organized by routes (e.g., /dashboard, /products, /orders) for intuitive navigation
    \item \textbf{Separation of Concerns}: Clear distinction between vendor and supplier interfaces with dedicated components and routes
    \item \textbf{Centralized Notifications}: Global toast notifications through Toaster components for system alerts and user feedback
\end{itemize}

Our routing structure demonstrates the application's organization:

\begin{lstlisting}[language=jsx, basicstyle=\small\ttfamily]
<Routes>
  <Route path="/" element={<Index />} />
  <Route path="/login" element={<Login />} />
  <Route path="/register" element={<Register />} />
  <Route path="/complete-profile" element={<CompleteProfile />} />
  <Route path="/dashboard" element={<Dashboard />} />
  <Route path="/products" element={<Products />} />
  <Route path="/orders" element={<Orders />} />
  <Route path="/suppliers" element={<Suppliers />} />
  <Route path="/settings" element={<Settings />} />
  <Route path="/forecast" element={<DemandForecast />} />
  <Route path="/route-optimization" element={<OrderAggregation />} />
  <Route path="*" element={<NotFound />} />
</Routes>
\end{lstlisting}

\subsection{Key UI Components}

The application is structured around several core screens that support the platform's functionality:

\begin{enumerate}
    \item \textbf{Dashboard (/dashboard)}: Central hub displaying key metrics, recent orders, and aggregation opportunities
    \item \textbf{Products (/products)}: Interface for vendors to browse and select products from suppliers
    \item \textbf{Orders (/orders)}: Comprehensive order management with status tracking and history
    \item \textbf{Suppliers (/suppliers)}: Directory of available suppliers with filtering capabilities
    \item \textbf{Demand Forecast (/forecast)}: Interactive visualization of product demand predictions
    \item \textbf{Order Aggregation (/order-aggregation)}: Interface for suppliers to manage aggregated orders and optimize delivery routes
\end{enumerate}

Each screen is designed with role-based access control, ensuring users only see functionality relevant to their role as either vendor or supplier.

\section{Integration and Communication}

\subsection{API Integration}

The frontend communicates with our FastAPI backend through React Query, providing efficient data fetching and caching:

\begin{itemize}
    \item \textbf{Centralized Query Client}: A shared QueryClientProvider wraps the application to manage all API requests
    \item \textbf{Typed API Responses}: TypeScript interfaces ensure type safety between frontend components and API data
    \item \textbf{Authentication Headers}: JWT tokens are automatically included in requests through Axios interceptors
    \item \textbf{Custom API Hooks}: Domain-specific React hooks encapsulate data fetching logic (e.g., useOrders, useProducts)
\end{itemize}

\subsection{Data Models and API Integration}

The ASTRO platform implements a comprehensive set of TypeScript interfaces to ensure type safety and data consistency across the application. These interfaces define the contract between frontend components and backend API services.

\subsubsection{Core Business Entity Interfaces}

We structured our data models around key business entities to ensure proper type checking and validation:

\begin{itemize}
    \item \textbf{User Models}: The system distinguishes between different user types using specialized interfaces:
          \begin{itemize}
              \item \texttt{UserBase} interface provides common user attributes (ID, name, contact information)
              \item \texttt{Vendor} and \texttt{Supplier} interfaces extend the base with role-specific properties
          \end{itemize}

    \item \textbf{Product Interface}: Represents product data including:
          \begin{itemize}
              \item Core product details (name, description, category)
              \item Inventory management fields (price, stock level, MOQ)
              \item Tracking metadata (supplier references, timestamps)
          \end{itemize}

    \item \textbf{Order Model}: Comprehensive order representation with:
          \begin{itemize}
              \item Order status tracking using a strongly-typed \texttt{OrderStatus} enum
              \item Nested \texttt{OrderItem} objects with product references and quantities
              \item Shipping details and payment information
          \end{itemize}
\end{itemize}

\subsubsection{Algorithm Response Interfaces}

The platform employs specialized interfaces for machine learning components:

\begin{itemize}
    \item \textbf{Demand Forecasting}: The \texttt{DemandForecast} interface structures time-series forecasting data:
          \begin{itemize}
              \item Sequential \texttt{ForecastPoint} objects with quantities and confidence intervals
              \item Trend analysis and direction indicators
              \item Model configuration parameters and performance metrics
          \end{itemize}

    \item \textbf{Order Aggregation}: The \texttt{OrderAggregationResponse} interface captures clustering results:
          \begin{itemize}
              \item \texttt{OrderAggregationCluster} objects with consolidated order information
              \item Geographic coordinates for delivery planning
              \item Algorithm performance metrics and configuration parameters
          \end{itemize}

    \item \textbf{Route Optimization}: The \texttt{RouteOptimizationResult} interface structures delivery routes:
          \begin{itemize}
              \item Sequenced waypoints with geographic coordinates
              \item Distance and time calculations between points
              \item Environmental data integration for adaptive routing
          \end{itemize}
\end{itemize}

\subsubsection{Dashboard Data Interfaces}

For the dashboard, specialized interfaces aggregate data for presentation:

\begin{itemize}
    \item \texttt{DashboardData}: Combines order statistics, supplier information, and financial metrics
    \item \texttt{OrderStatusCount}: Provides a breakdown of orders by current status
    \item \texttt{MonthlyOrderData}: Tracks order volume trends for visualization
    \item \texttt{StatCardProps} and other component-specific interfaces for UI rendering
\end{itemize}

These strongly-typed interfaces ensure data consistency throughout the application while providing robust documentation and autocomplete support during development. The interface-driven approach facilitates seamless integration between frontend components and backend services by establishing clear contracts for data exchange.
The notification system is particularly important for the order aggregation feature, as it alerts vendors about potential bulk ordering opportunities with nearby businesses, enabling real-time collaboration that would otherwise be difficult to coordinate.
\section{Summary}

The implementation of the ASTRO platform leverages modern technologies and architecture patterns to deliver a robust, scalable, and performant solution. By utilizing FastAPI for the backend and Flutter for the frontend, the system achieves a balance between development efficiency and runtime performance. The modular design ensures that each component order aggregation, logistics optimization and demand forecasting can evolve independently while maintaining integration with the broader system.

The implementation demonstrates how complex algorithms and advanced machine learning techniques can be effectively deployed in a production environment, providing tangible benefits to vendors through improved demand forecasting, efficient order aggregation, and optimized logistics.